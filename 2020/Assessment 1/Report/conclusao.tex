\chapter{Conclusão}
Para a resolução deste problema, a modulação ocorreu de um modo bastante natural, partindo do mesmo como um problema de minimização, e chegamos rapidamente a uma função objetivo que satisfazia os nossos critérios, usando a tabela de distâncias entre as arestas. 

Após conseguir construir todas as restrições necessárias para obter soluções aceitáveis, obtivemos uma modelação completa, e apenas necessitamos de aplicar a mesma ao \emph{lpsolve} para obter uma solução ótima para o problema.

Este trabalho ajudou-nos a aprofundar o nosso uso de \emph{solvers} para estes problemas, e mostrou-nos também a importância da modelação acima de tudo, e de conseguir transformar um problema em funções e restrições que o descrevem concretamente e sem perder nenhuma informação, de modo a conseguir obter soluções ótimas para problemas mais complexos.