\chapter{Introdução}

Para este primeiro projeto, desenvolvido no âmbito da unidade curricular de Modelos Determinísticos de Investigação Operacional, é apresentado um problema comum de minimização, em que se pretende encontrar um percurso ótimo que percorra todo um conjunto de pontos ou arestas, minimizando a distância total percorrida. 

Mais especificamente, temos o caso de um veículo não tripulado com necessidade de verificar linhas de energia eléctrica em alta tensão para verificar se há vegetação a interferir com as linhas. Sendo este veículo um drone, é possível fazer reposicionamentos entre vértices sem ter de seguir arestas predeterminadas, correspondentes a um deslocamento aéreo entre pontos seguindo o trajeto mais curto, mas mantém-se a necessidade de verificar todas as linhas, pelo menos uma vez.

De modo a apresentar uma solução a este problema, procedemos a modelá-lo, de modo a poder descrevê-lo de um modo mais facilmente compreensível e objetivo, permitindo-nos seguidamente aplicar métodos de programação linear, com o auxílio de software, para encontrar uma solução ótima para o problema.